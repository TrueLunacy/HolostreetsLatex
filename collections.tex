\begin{luacode*}
function outputLineCommand(name, commandSuffix, numOfArgs, argLeft, argRight)
    local t = {}
    for x=1,numOfArgs do
        t[x] = argLeft .. "#" .. x .. argRight
    end
    local outputline = table.concat(t, " & ") .. "\\\\"
    local str = "\\makeatletter\\newcommand{\\" .. name .. commandSuffix .. "}[" .. numOfArgs .. "]{"
        .. "\\immediate\\write\\@auxout{\\detokenize{\\@writefile{" .. name .. ".table}{"
        .. outputline
        .. "}}}}\\makeatother"

    tex.sprint(str)
end
\end{luacode*}

% first - name
% second - how many arguments
% third - tabular column specifier
% fourth - first row (for column names)
\newcommand{\newcollection}[3]
{
    \immediate\write\@auxout{\detokenize{\@writefile{#1.table}{\begin{tabularx}{\textwidth}{#3}}}}
    \AtEndDocument{\immediate\write\@auxout{\detokenize{\@writefile{#1.table}{\end{tabularx}}}}}

    \luadirect{outputLineCommand("#1", "", #2, "", "")}
    \luadirect{outputLineCommand("#1", "Header", #2, "\string\\tableheader{", "}")}
    \expandafter\newcommand\csname #1RowDirect\endcsname[1]
    {
        \immediate\write\@auxout{\detokenize{\@writefile{#1.table}{##1 \\}}}
    }
    \expandafter\newcommand\csname #1WholeRow\endcsname[1]
    {
        \immediate\write\@auxout{\detokenize{\@writefile{#1.table}{\wholerow{#2}{##1} \\}}}
    }

    \expandafter\newcommand\csname #1TableOut\endcsname
    {
        \@starttoc{#1.table}
    }
}